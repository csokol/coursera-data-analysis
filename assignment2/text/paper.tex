\documentclass[IEEEtran]{IEEEtran}
\usepackage[utf8]{inputenc}
\usepackage{flushend}
\usepackage{listings}
\usepackage{hyperref} 
\usepackage{verbatim} 
\usepackage[pdftex]{graphicx} 
\usepackage[T1]{fontenc}
\usepackage[table]{xcolor}
\usepackage{soul}
\graphicspath{{./images/}}
\urlstyle{same} 

\usepackage{fancyvrb}
\usepackage{color}



\makeatletter
\def\PY@reset{\let\PY@it=\relax \let\PY@bf=\relax%
    \let\PY@ul=\relax \let\PY@tc=\relax%
    \let\PY@bc=\relax \let\PY@ff=\relax}
\def\PY@tok#1{\csname PY@tok@#1\endcsname}
\def\PY@toks#1+{\ifx\relax#1\empty\else%
    \PY@tok{#1}\expandafter\PY@toks\fi}
\def\PY@do#1{\PY@bc{\PY@tc{\PY@ul{%
    \PY@it{\PY@bf{\PY@ff{#1}}}}}}}
\def\PY#1#2{\PY@reset\PY@toks#1+\relax+\PY@do{#2}}

\expandafter\def\csname PY@tok@gu\endcsname{\let\PY@bf=\textbf}
\expandafter\def\csname PY@tok@s2\endcsname{\def\PY@tc##1{\textcolor[rgb]{0.16,0.00,1.00}{##1}}}
\expandafter\def\csname PY@tok@gs\endcsname{\let\PY@bf=\textbf}
\expandafter\def\csname PY@tok@cm\endcsname{\def\PY@tc##1{\textcolor[rgb]{0.25,0.37,0.75}{##1}}}
\expandafter\def\csname PY@tok@gp\endcsname{\let\PY@bf=\textbf}
\expandafter\def\csname PY@tok@m\endcsname{\def\PY@tc##1{\textcolor[rgb]{0.00,0.00,0.00}{##1}}}
\expandafter\def\csname PY@tok@mh\endcsname{\def\PY@tc##1{\textcolor[rgb]{0.00,0.00,0.00}{##1}}}
\expandafter\def\csname PY@tok@ge\endcsname{\let\PY@it=\textit}
\expandafter\def\csname PY@tok@il\endcsname{\def\PY@tc##1{\textcolor[rgb]{0.00,0.00,0.00}{##1}}}
\expandafter\def\csname PY@tok@cs\endcsname{\def\PY@tc##1{\textcolor[rgb]{0.25,0.50,0.37}{##1}}}
\expandafter\def\csname PY@tok@cp\endcsname{\let\PY@it=\textit\def\PY@tc##1{\textcolor[rgb]{0.38,0.38,0.38}{##1}}}
\expandafter\def\csname PY@tok@gh\endcsname{\let\PY@bf=\textbf}
\expandafter\def\csname PY@tok@ni\endcsname{\def\PY@tc##1{\textcolor[rgb]{0.00,0.00,0.00}{##1}}}
\expandafter\def\csname PY@tok@nn\endcsname{\def\PY@tc##1{\textcolor[rgb]{0.00,0.00,0.00}{##1}}}
\expandafter\def\csname PY@tok@na\endcsname{\def\PY@tc##1{\textcolor[rgb]{0.00,0.00,0.00}{##1}}}
\expandafter\def\csname PY@tok@s1\endcsname{\def\PY@tc##1{\textcolor[rgb]{0.16,0.00,1.00}{##1}}}
\expandafter\def\csname PY@tok@nc\endcsname{\def\PY@tc##1{\textcolor[rgb]{0.00,0.00,0.00}{##1}}}
\expandafter\def\csname PY@tok@nd\endcsname{\def\PY@tc##1{\textcolor[rgb]{0.39,0.39,0.39}{##1}}}
\expandafter\def\csname PY@tok@ne\endcsname{\def\PY@tc##1{\textcolor[rgb]{0.00,0.00,0.00}{##1}}}
\expandafter\def\csname PY@tok@nf\endcsname{\def\PY@tc##1{\textcolor[rgb]{0.00,0.00,0.00}{##1}}}
\expandafter\def\csname PY@tok@si\endcsname{\def\PY@tc##1{\textcolor[rgb]{0.16,0.00,1.00}{##1}}}
\expandafter\def\csname PY@tok@sh\endcsname{\def\PY@tc##1{\textcolor[rgb]{0.16,0.00,1.00}{##1}}}
\expandafter\def\csname PY@tok@nt\endcsname{\let\PY@bf=\textbf\def\PY@tc##1{\textcolor[rgb]{0.50,0.00,0.33}{##1}}}
\expandafter\def\csname PY@tok@ow\endcsname{\def\PY@tc##1{\textcolor[rgb]{0.00,0.00,0.00}{##1}}}
\expandafter\def\csname PY@tok@sx\endcsname{\def\PY@tc##1{\textcolor[rgb]{0.16,0.00,1.00}{##1}}}
\expandafter\def\csname PY@tok@c1\endcsname{\def\PY@tc##1{\textcolor[rgb]{0.25,0.50,0.37}{##1}}}
\expandafter\def\csname PY@tok@kc\endcsname{\let\PY@bf=\textbf\def\PY@tc##1{\textcolor[rgb]{0.50,0.00,0.33}{##1}}}
\expandafter\def\csname PY@tok@c\endcsname{\def\PY@tc##1{\textcolor[rgb]{0.25,0.50,0.37}{##1}}}
\expandafter\def\csname PY@tok@mf\endcsname{\def\PY@tc##1{\textcolor[rgb]{0.00,0.00,0.00}{##1}}}
\expandafter\def\csname PY@tok@err\endcsname{\def\PY@bc##1{\setlength{\fboxsep}{0pt}\fcolorbox[rgb]{1.00,0.00,0.00}{1,1,1}{\strut ##1}}}
\expandafter\def\csname PY@tok@kd\endcsname{\let\PY@bf=\textbf\def\PY@tc##1{\textcolor[rgb]{0.50,0.00,0.33}{##1}}}
\expandafter\def\csname PY@tok@ss\endcsname{\def\PY@tc##1{\textcolor[rgb]{0.16,0.00,1.00}{##1}}}
\expandafter\def\csname PY@tok@sr\endcsname{\def\PY@tc##1{\textcolor[rgb]{0.16,0.00,1.00}{##1}}}
\expandafter\def\csname PY@tok@mo\endcsname{\def\PY@tc##1{\textcolor[rgb]{0.00,0.00,0.00}{##1}}}
\expandafter\def\csname PY@tok@mi\endcsname{\def\PY@tc##1{\textcolor[rgb]{0.00,0.00,0.00}{##1}}}
\expandafter\def\csname PY@tok@kn\endcsname{\let\PY@bf=\textbf\def\PY@tc##1{\textcolor[rgb]{0.50,0.00,0.33}{##1}}}
\expandafter\def\csname PY@tok@o\endcsname{\def\PY@tc##1{\textcolor[rgb]{0.00,0.00,0.00}{##1}}}
\expandafter\def\csname PY@tok@kr\endcsname{\let\PY@bf=\textbf\def\PY@tc##1{\textcolor[rgb]{0.50,0.00,0.33}{##1}}}
\expandafter\def\csname PY@tok@s\endcsname{\def\PY@tc##1{\textcolor[rgb]{0.16,0.00,1.00}{##1}}}
\expandafter\def\csname PY@tok@kp\endcsname{\let\PY@bf=\textbf\def\PY@tc##1{\textcolor[rgb]{0.94,0.00,0.00}{##1}}}
\expandafter\def\csname PY@tok@w\endcsname{\def\PY@tc##1{\textcolor[rgb]{0.73,0.73,0.73}{##1}}}
\expandafter\def\csname PY@tok@kt\endcsname{\let\PY@bf=\textbf\def\PY@tc##1{\textcolor[rgb]{0.50,0.00,0.33}{##1}}}
\expandafter\def\csname PY@tok@sc\endcsname{\def\PY@tc##1{\textcolor[rgb]{0.16,0.00,1.00}{##1}}}
\expandafter\def\csname PY@tok@sb\endcsname{\def\PY@tc##1{\textcolor[rgb]{0.16,0.00,1.00}{##1}}}
\expandafter\def\csname PY@tok@k\endcsname{\let\PY@bf=\textbf\def\PY@tc##1{\textcolor[rgb]{0.50,0.00,0.33}{##1}}}
\expandafter\def\csname PY@tok@se\endcsname{\def\PY@tc##1{\textcolor[rgb]{0.16,0.00,1.00}{##1}}}
\expandafter\def\csname PY@tok@sd\endcsname{\def\PY@tc##1{\textcolor[rgb]{0.16,0.00,1.00}{##1}}}

\def\PYZbs{\char`\\}
\def\PYZus{\char`\_}
\def\PYZob{\char`\{}
\def\PYZcb{\char`\}}
\def\PYZca{\char`\^}
\def\PYZam{\char`\&}
\def\PYZlt{\char`\<}
\def\PYZgt{\char`\>}
\def\PYZsh{\char`\#}
\def\PYZpc{\char`\%}
\def\PYZdl{\char`\$}
\def\PYZti{\char`\~}
% for compatibility with earlier versions
\def\PYZat{@}
\def\PYZlb{[}
\def\PYZrb{]}

\begin{document}

\title{Data Analysis Project 2}
\author{Francisco Sokol}

\maketitle

\IEEEpeerreviewmaketitle

\section{Introduction}

In this project, we were challenged to develop a function to predict the
activity of a person according to data collected by a smartphone. Modern
smartphones can measure the user's movement patterns through the data captured
by a accelerometer bundled with the phone \cite{acelerometro}. So data collected by samsung
smartphones accelerometers was provided to us, labeled with the activity of the
person using the smartphone.

\section{Method}

The data was collected through Samsung smartphones and can be found here:
\url{https://spark-public.s3.amazonaws.com/dataanalysis/samsungData.rda}. In
this format, the data is ready to be loaded to R environment. Since the data
was already processed, it was'nt necessary to clean the data.

We spplited the data into to data sets, the training set and the test set. The
training set contained all rows with the value of column ``subject'' less than
26, while the test contained the rest of the data. So the size of the training
set was 3965 rows, while the test set was 1485.

The data set provided contained 561 columns of quantitative measurements
collected by accelerometers. We designed some exploratory graphs to better
understand the relation between the variable and the associated actities.

Since it was hard to analyze which columns were more significant to describe
the human activity associated, we decided to use all the rows to create a model
for classificating the data.

To develop the model to predict the human activities, we used the Random Forest
R package and the randomForest function to build the predictive model with the
training set. This function creates a predictive model by randomly calculating
various classification trees and later using them to predict the outcome
\cite{random-forest}.

The following code 

\begin{Verbatim}[commandchars=\\\{\}]
library(tree)
library(randomForest)

load("data/samsungData.rda")

set.seed(10293812)

trainingSet = samsungData[samsungData\PYZdl{}subject \PYZlt{}= 25,] 
trainingSet = data.frame(trainingSet)
trainingSet\PYZdl{}activity = as.factor(trainingSet\PYZdl{}activity)

testSet = samsungData[samsungData\PYZdl{}subject \PYZgt{}= 27,]
testSet = data.frame(testSet)
testSet\PYZdl{}activity = as.factor(testSet\PYZdl{}activity)

forest \PYZlt{}- randomForest(activity \PYZti{} . -activity, 
                      data = trainingSet, prox=TRUE)
predictions = predict(forest, testSet)
testSetSize = length(predictions)
correctPredictions = predictions == testSet\PYZdl{}activity

errors = table(correctPredictions)[1]
misclassRate = errors/testSetSize

plot(forest)
\end{Verbatim}


After running the randomForest function, we could use the resulting object to
make predictions from the test data set. 


\section{Results}








\begin{thebibliography}{24}

\bibitem{random-forest} Random Forest package site: \url{http://www.stat.berkeley.edu/~breiman/RandomForests/cc_home.htm}
\bibitem{acelerometro} Wikipedia article about accelerometer: \url{http://en.wikipedia.org/wiki/Accelerometer}

\bibitem{murphy} Murphy-Hill, E.; Parnin, C.; Black, A. P  How we refactor, and how we know it. 
Proceedings of the 2009 IEEE 31st International Conference on Software Engineering, 2009.

\end{thebibliography}

% that's all folks
\end{document}

